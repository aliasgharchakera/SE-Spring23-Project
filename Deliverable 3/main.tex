\documentclass{article}
%%%%%%% PACKAGES %%%%%%%%
\usepackage[utf8]{inputenc}
\usepackage[margin=3cm]{geometry}

\title{Software Engineering Project Technical Constraints}
\author{Ali Asghar Yousuf $\mid$ Muhammad Azeem Haider\\
        Muhammad Shahid Mehmood $\mid$ Musab Sattar}
\date{\today}

\begin{document}

\maketitle

\section*{Architecture Decisions}

Thekedaar Hatao is a unique mobile application with the aim to remove the
middleman (thekedaar) during the construction process. The application will
have quite a number of technical features that will be addressed in the
subsequent sections. The application is being made using two professional
frameworks.

\begin{itemize}

        \item Flutter
        \item Django

\end{itemize}

Lets dive into our reasoning for choosing Flutter and Django.

\subsubsection*{Flutter}

Flutter is a high quality framework that makes it easy to build a mobile
application swiftly. Since the application will be compatible with both Android
and IOS, Flutter seems like the obvious and the best choice for our Frontend
development. Flutter can run in any browser, which is why it provides us with
the chance to work with Django without creating many hurdles.

\subsubsection*{Django}

Django was an executive decision that the group members all agreed on. The plan
was to first use firebase for Backend development but since we want to work
with the Database and the server, Django seems like a good fit for the
application and the team. Django when combined with Flutter offers some
excellent results when developing a mobile application.

\section*{Compatibility}

We want to keep our mobile application simple and efficient. For this purpose,
while we will not be adding any flashy feature, we will want our application to
run smoothly. For this purpose we would advice the users to use a mobile phone
with at least Android 10 or iOS 15 for optimum performance. The application will
be supported by both the Operating systems and tested on both the operating
systems to make sure that users at the time of release face as little
problems as possible.

\section*{Application Features}

The mobile application idea for Thekedaar Hatao is simple yet elegant. The
application will have three major features that are as followed;

\begin{itemize}

        \item Built in calculator
        \item Forum to interact with other users
        \item Marketplace to Buy and Sell

\end{itemize}

\subsubsection*{Built in Calculator}

The idea for the built in calculator stems from the frustration that people go
through when the middleman provides them with bad advice prompting them to buy
more material than first needed. The calculator will be added which will ask
the user to add the land in sq/ft, and how many floors one is planning on
building. Once the user adds the sq/ft of land and the floor that they are
planning on building, the calculator will provide them with an estimate raw
material amount needed. This will include cement, cinder block, sand and other
such requirements needed to build a house.

% Can also add the cost of this material 

\subsubsection*{Forum}

Forum will be a place for advice where the users can ask and advice others on
problems that have arisen. The forum will be there for users to connect with
each other and will provide an option to talk in private with each other as
well.

\subsubsection*{MarketPlace}

The marketplace will be available on the application for users to sell the
extra material left once their house is finished. This will also be beneficial
for people looking for small amounts of material to buy for little construction
around the house or if the construction is almost complete and only some amount
of material is needed.

\end{document}