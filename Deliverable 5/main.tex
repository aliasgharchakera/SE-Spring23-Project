\documentclass{article}
%%%%%%% PACKAGES %%%%%%%%
\usepackage[utf8]{inputenc}
\usepackage{graphicx}
\usepackage{geometry}
\geometry{a4paper, margin=1in}
\usepackage{hyperref}
\hypersetup{
    colorlinks=true,
    linkcolor=blue,
    filecolor=magenta,      
    urlcolor=cyan,
}

\title{Software Engineering Project Architecture}
\author{Muhammad Azeem Haider $\mid$ Ali Asghar Yousuf \\
      Muhammad Shahid Mehmood $\mid$ Musab Sattar}
\date{\today}

\begin{document}

\maketitle

\section{Introduction}

THekeydaar hatao is an app that gives an estimate of the amount of raw material that might be needed to create your house based on how much land the house is to be built on. It provides several functions such as a login, signup, forum to communicate on, material estimator, and marketplace.

\section{Authentication}

\subsection{POST /signup}

\subsubsection{Description}

This endpoint allows users to create a new account on the app.

\subsubsection{Parameters}

\begin{itemize}
\item email (required) - The email address of the user.
\item password (required) - The password for the user's account.
\item name (optional) - The name of the user.
\end{itemize}

\subsubsection{Example Request}

\begin{verbatim}
POST /signup
{
    "email": "example@example.com",
    "password": "password123",
    "name": "Babar Azam"
}
\end{verbatim}

\subsubsection{Example Response}

\begin{verbatim}
{
    "user_id": 123,
    "message": "Account created successfully"
}
\end{verbatim}

\subsection{POST /login}

\subsubsection{Description}

This endpoint allows users to log in to their account on the app.

\subsubsection{Parameters}

\begin{itemize}
\item email (required) - The email address of the user.
\item password (required) - The password for the user's account.
\end{itemize}

\subsubsection{Example Request}

\begin{verbatim}
POST /login
{
    "email": "example@example.com",
    "password": "password123"
}
\end{verbatim}

\subsubsection{Example Response}

\begin{verbatim}
{
    "user_id": 123,
    "message": "Login successful"
}
\end{verbatim}

If the email or password is incorrect, the API should return a 401 Unauthorized response with an appropriate error message.

\section{Material Estimator}

\subsection{GET /estimate}

\subsubsection{Description}

This endpoint allows users to estimate the raw material required to build a house based on the area of land.

\subsubsection{Parameters}

\begin{itemize}
  \item area (required) - The area of land in square feet.
\end{itemize}

\subsubsection{Example Request}

\begin{verbatim}
GET /estimate?area=1000
\end{verbatim}

\subsubsection{Example Response}

\begin{verbatim}
{
    "materials": {
        "bricks": 5000,
        "cement": 20,
        "sand": 1000,
        "steel": 500,
        "wood": 100
    },
    "cost": 45000
}
\end{verbatim}

\section{Forum}

\subsection{GET /forum}

\subsubsection{Description}

This endpoint allows users to retrieve a list of threads posted on the forum.

\subsubsection{Parameters}

\begin{itemize}
\item limit (optional) - The maximum number of threads to retrieve. Default is 10.
\item offset (optional) - The offset from which to start retrieving threads. Default is 0.
\end{itemize}

\subsubsection{Example Request}

\begin{verbatim}
GET /forum?limit=20&offset=10
\end{verbatim}

\subsubsection{Example Response}

\begin{verbatim}
{
    "threads": [
        {
        "id": 1,
        "title": "Need advice on flooring",
        "author": "Babar Azam",
        "created_at": "2022-01-01T12:00:00Z",
        "last_post_at": "2022-01-02T08:00:00Z",
        "replies_count": 5
        },
        {
        "id": 2,
        "title": "Kitchen cabinet recommendations",
        "author": "Azam Khan",
        "created_at": "2022-01-03T10:00:00Z",
        "last_post_at": "2022-01-04T15:00:00Z",
        "replies_count": 3
        }
    ]
}
\end{verbatim}

\subsection{POST /forum}

\subsubsection{Description}

This endpoint allows users to create a new thread on the forum.

\subsubsection{Parameters}

\begin{itemize}
\item title (required) - The title of the thread.
\item author (required) - The name of the thread author.
\item content (required) - The content of the first post in the thread.
\end{itemize}

\subsubsection{Example Request}

\begin{verbatim}
POST /forum
{
    "title": "Need advice on roofing",
    "author": "Babar Azam",
    "content": "I'm planning to build a new house and I'm not sure what type of roofing to choose. Any recommendations?"
}
\end{verbatim}

\subsubsection{Example Response}

\begin{verbatim}
{
    "id": 3,
    "title": "Need advice on roofing",
    "author": "Babar Azam",
    "created_at": "2022-01-05T09:00:00Z",
    "last_post_at": "2022-01-05T09:00:00Z",
    "replies_count": 0
}
\end{verbatim}

\section{Personal Chat}

\subsection{GET /chat}

\subsubsection{Description}

This endpoint retrieves the user's list of conversations.

\subsubsection{Parameters}

\begin{itemize}
    \item user\_ID (required) - The id of the user.
\end{itemize}

\subsubsection{Example Request}

\begin{verbatim}
GET /chat?user_id=123
\end{verbatim}

\subsubsection{Example Response}

\begin{verbatim}
{
"conversations": [
        {
        "id": 1,
        "name": "Babar Azam",
        "unread_count": 2,
        "last_message": "Hey, how are you doing?"
        },
        {
        "id": 2,
        "name": "Azam Khan",
        "unread_count": 0,
        "last_message": "See you tomorrow at 5 pm"
        }
    ]
}
\end{verbatim}

\subsection{GET /chat/:id}

\subsubsection{Description}

This endpoint retrieves the conversation between the user and another user.

\subsubsection{Parameters}

\begin{itemize}
    \item user\_id (required) - The ID of the user.
    \item id (required) - The ID of the conversation.
\end{itemize}

\subsubsection{Example Request}

\begin{verbatim}
GET /chat/2?user_id=123
\end{verbatim}

\subsubsection{Example Response}

\begin{verbatim}
{
"id": 2,
"messages": [
        {
        "id": 1,
        "sender_id": 123,
        "receiver_id": 456,
        "content": "Hey, how are you doing?",
        "timestamp": "2023-03-04T15:00:00Z"
        },
        {
        "id": 2,
        "sender_id": 456,
        "receiver_id": 123,
        "content": "I'm good, thanks! How about you?",
        "timestamp": "2023-03-04T15:05:00Z"
        },
        {
        "id": 3,
        "sender_id": 123,
        "receiver_id": 456,
        "content": "I'm doing well too, thanks for asking.",
        "timestamp": "2023-03-04T15:10:00Z"
        }
    ]
}
\end{verbatim}

\end{document}