\documentclass{article}
%%%%%%% PACKAGES %%%%%%%%
\usepackage[utf8]{inputenc}
\usepackage[margin=3cm]{geometry}

\title{Software Engineering Project Proposal}
\author{Muhammad Azeem Haider $\mid$ Ali Asghar Yousuf \\
        Muhammad Shahid Mehmood $\mid$ Musab Sattar}
\date{\today}

\begin{document}

\maketitle

\section*{Project Title}

\textbf{Thekedaar Hatao}

\section*{Project Team Name}

\textbf{Babar Azam Hamood CS}

\section*{Project Team Member}

\begin{enumerate}
    \item Muhammad Azeem Haider
    \item Ali Asghar Yusuf
    \item Muhammad Shahid Mehmood
    \item Musab Sattar
\end{enumerate}

\section*{The Problem}

Building your dream house in Pakistan can be a difficult and stressful process. While material costs are at an all time high, the middle man (Thekedaar) poses
potential problems itself. Most times, prompting you to buy more material than you need or giving you wrong advice, just for Thekedaar own financial benefit.
This problem has raged on for so long that people are fearful of getting their dream houses built. With no proper advice in the local context when it comes to
making houses in Pakistan, thaikaydaar hatao aims to help you build your house that you could live and propser in. 

\section*{Proposed Solution}

Thekedaar hatao is an application that calculates the amount of material you may need depending on how much land in square feet, you plan on building on.
The calculator on the app, will provide you with an estimated amount of cement, bricks, sand, and other such items that are needed while building your house.
In addition, the application will help cater the dreamers by having a forum where people can talk about the problems that have arisen and how they tackled with them.
Advice threads and asking for help will be done on the forum. Since you can never be sure of how much material you exactly need, there is always a (+ or -)5\%
chance of having extra material, you can also sell this to other users on the app through a buy and sell section.

\section*{Goals and Objectives}

\begin{itemize}
    \item To make the building of your house easier
    \item Saving cost by calculating exactly how much material you need for your house
    \item Saving money by removing the hassle of a middle man
    \item Connecting different people building their house on the internet
    \item Making a builder feel as if they are part of a community and not alone when building their house
    \item Making the application as simple and easy to use as possible
\end{itemize}

\section*{Features Covered}
\begin{itemize}
    \item Calculator to give you estimated material required depending on square feet land
    \item Forum where members of the community can communicate with each other and can get quick help with their specific issues.
    \item Marketplace where sellers can market their goods and users can find reputable vendors
    \item Private chat so that users can work out personal deals with vendors
\end{itemize}


\section*{Technical Solution Overview}

The frontend of the application needs to be simple and efficient, for this purpose we will be using \textbf{flutter}, for the backend purposes, we will be using
\textbf{Firebase} which will help us store our relevant information and user data. For design purposes we will use \textbf{Figma}. And finally for communication
we will be using \textbf{Jira} and \textbf{Slack}. 

\end{document}